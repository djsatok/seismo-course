\documentclass[12pt]{article}

\usepackage[utf8]{inputenc}
\usepackage[english,russian]{babel}
\usepackage{amsmath, amssymb}
\usepackage{graphicx}
\usepackage{epstopdf}
\usepackage{algorithm}
\usepackage{algpseudocode}

\newcommand{\cutefrac}[2]{{}^{#1\!}\!/{\!}_#2}
\newcommand{\half}{\cutefrac{1}{2}}
\newcommand{\cuteddots}{
\text{
\raisebox{.6ex}{$\cdot$}\hspace{.75em}%
\raisebox{.3ex}{$\cdot$}\hspace{.75em}%
$\cdot$\hspace{.75em}%
\raisebox{-.3ex}{$\cdot$}\hspace{.75em}%
\raisebox{-.6ex}{$\cdot$}%
}}

\graphicspath{{mp/}}

\title{Задание 2}
\date{}

\begin{document}

\maketitle

Решение задачи
\[
\begin{cases}
p_{tt} = c^2(x) p_{xx}\\
p\big|_{t = 0} = 0\\
p_t\big|_{t = 0} = 0\\
p\big|_{x = 0} = \sin \omega t\\
p_t + c(1) p_x \big|_{x = 1} = 0
\end{cases},
\]
довольно быстро становится периодическим во времени:
\begin{equation}
p = \operatorname{Im} \left[e^{i \omega t} u(x)\right] = 
\operatorname{Re} u(x) \sin \omega t +
\operatorname{Im} u(x) \cos \omega t
\label{eq:harm}
\end{equation}
где $u(x)$ --- комплексная амплитуда колебаний в точке $x$.

Подстановкой \eqref{eq:harm} в исходное уравнение, получаем:
\begin{equation}
\begin{cases}
-\omega^2 u = c^2(x) u_{xx}\\
u(0) = 1\\
i\omega u(1) + c(1) u_x(1) = 0
\end{cases}.
\label{eq:stationary}
\end{equation}
При этом начальные условия были отброшены, поскольку мы ищем стационарное
решение на $t \rightarrow \infty$. Если бы в \eqref{eq:harm} вместо 
$\operatorname{Im}$ использовалось $\operatorname{Re}$, то это повлияло бы на
граничное условие для $u(0)$.

Отметим, что \eqref{eq:stationary} можно записать в форме, сходной со
стационарным уравнением Шрёдингера. Это позволить переносить результаты из
теории уравнения Шрёдингера непосредственно на нашу задачу:
\[
\begin{cases}
- \Delta u + V(x) u = E u, \qquad V(x) =
-\dfrac{\omega^2}{c^2(x)}, \quad E = 0\\
u(0) = 1\\
u_x(1) = -\dfrac{i\omega}{c(1)} u(1)
\end{cases}.
\label{eq:shroedinger}
\]

Запишем задачу в слабой постановке, интегрируя с функцией $w(x)$:
\begin{multline}
0 = \int_0^1 \left[-u_{xx}(x) - \frac{\omega^2}{c^2(x)} u(x)\right] w(x) dx = \\
=-u_x w\Big|_0^1 + \int_0^1 u_x(x) w_x(x) dx - \omega^2 \int_0^1 \frac{u(x)
w(x)}{c^2(x)} dx
\end{multline}
Далее, следуя уже известной процедуре, разложим $u(x)$ по базису $\psi_i(x)$:
\[
u(x) = \sum_{k=0}^N u_k \psi_k(x).
\]
Граничное условие в точке $x = 0$ позволяет сразу же определить $u_0 = 1$.
Это жесткое граничное условие, и, следовательно для проверочной функции $w(x)$ нужно
поставить граничное условие $w(0) = 0$. Разложение $w(x)$ имеет вид:
\[
w(x) = \sum_{j=1}^N w_j \psi_j(x).
\]
Из-за линейности, в качестве $w(x)$ достаточно взять все $\psi_j(x),\; j = 1,
\dots, N$. Подставляя $w(0) = 0, u_x(1) = -\dfrac{i\omega}{c(1)} u(1)$, получаем:
\begin{multline}
0 = -u_x w\Big|_0^1 + \int_0^1 u_x(x) w_x(x) dx - \omega^2 \int_0^1
\frac{u(x)w(x)}{c^2(x)}dx = \\ 
=
i\omega \frac{u(1) w(1)}{c(1)}
+ \int_0^1 u_x(x) w_x(x) dx 
- \omega^2 \int_0^1 \frac{u(x)w(x)}{c^2(x)} dx.
\end{multline}
Подставляя вместо функций их разложения по базису, получаем систему линейных
уравнений:
\begin{multline}
i\omega \sum_{k=0}^N u_k \frac{\psi_k(1)\psi_j(1)}{c(1)}
+\sum_{k=0}^N u_k \int_0^1 \psi'_k(x) \psi'_j(x) dx
-\omega^2 \sum_{k=0}^N u_k \int_0^1 \frac{\psi_k(x) \psi_j(x)}{c^2(x)} dx = \\
= \sum_{k=0}^N u_k \left(i\omega D_{kj} + K_{kj} -\omega^2 M_{kj}\right) = 0.
\end{multline}
Здесь матрицы $K, M$ и $D$ --- те же, что и в задании 1. Исключим из системы
известное значение $u_0 = 1$:
\[
\sum_{k=1}^N u_k \left(i\omega D_{kj} + K_{kj} -\omega^2 M_{kj}\right) =
\omega^2 M_{0j} - K_{0j} -i\omega D_{0j} \equiv f_j, \qquad j = 1, \dots, N
\]
или в матричной форме
\begin{gather*}
\mathbf{u}^T \left(-\omega^2 \mathbf{M} + i\omega \mathbf{D} + \mathbf{K}\right) =
\mathbf{f}^T\\
\mathbf{M} = 
\frac{h}{6}
\begin{pmatrix}
2\gamma_{\half} + 2\gamma_{\cutefrac{3}{2}} & \gamma_{\cutefrac{3}{2}}\\
\gamma_{\cutefrac{3}{2}} & 2\gamma_{\cutefrac{3}{2}} + 2\gamma_{\cutefrac{5}{2}} & \gamma_{\cutefrac{5}{2}}\\
& \cuteddots\; & \cuteddots & \;\cuteddots \\
& & \gamma_{N - \cutefrac{3}{2}} & 2\gamma_{N - \cutefrac{3}{2}} + 2\gamma_{N
- \half} & \gamma_{N-\half}\\
& & & \gamma_{N - \half} & 2\gamma_{N-\half}
\end{pmatrix}
\\
\mathbf{K} = 
\frac{1}{h}
\begin{pmatrix}
2 & -1\\
-1 & 2 & -1\\
& \ddots & \ddots & \ddots \\
& & -1 & 2 & -1\\
& & & -1 & 1
\end{pmatrix}\qquad
\mathbf{D} = 
\begin{pmatrix}
0  \\
& 0 \\
& & \ddots \\
& & & 0 \\
& & & & \frac{1}{c(1)}
\end{pmatrix}\\
\mathbf{u}^T = \begin{pmatrix} u_1 & u_2 & \cdots & u_N \end{pmatrix}\\
\mathbf{f}^T = \begin{pmatrix} \dfrac{\omega^2 h}{6}\gamma_{\half} + \dfrac{1}{h} & 0 & \cdots & 0 \end{pmatrix}
\end{gather*}
Для удобства транспонируем систему (матрицы симметричны):
\[
\left(-\omega^2 \mathbf{M} + i\omega \mathbf{D} + \mathbf{K}\right)\mathbf{u} =
\mathbf{f}.\\
\]
Данная система является трехдиагональной системой линейных уравнений
относительно неизвестных $u_1, \dots, u_N$. Коэффициенты матрицы содержат
комплексные элементы.
\[
\begin{pmatrix}
b_1 & a_2 \\
a_2 & b_2 & a_3 \\
& a_3 & b_3 & a_4 \\
&& \ddots & \ddots & \ddots\\
&&& a_{N-1} & b_{N-1} & a_N\\
&&&& a_N & b_N
\end{pmatrix}
\begin{pmatrix}
u_1\\u_2\\u_3\\\vdots\\u_{N-1}\\u_N
\end{pmatrix} = 
\begin{pmatrix}
d_1\\d_2\\d_3\\\vdots\\d_{N-1}\\d_N
\end{pmatrix}
\]

\section{Задание}

Численно найти решение системы уравнений методом прогонки или редукции.
Использовать следующие параметры: 
\[N = 200,\quad \omega = 5,\quad T = 10,\quad c(x) = 0.1 + 3.6 (x - 0.5)^2\]
В качестве решения представить графики функций $\operatorname{Re} u(x), \operatorname{Im}
u(x)$ и функцию $\operatorname{Im}\left[u(x) \exp^{i\omega T}\right]$.

\section{Приближенное решение}

Чтобы получить представление о приближенном решении, воспользуемся из
результатов теории квазиклассического приближения для уравнение Шрёдингера:
\[
\begin{cases}
-u_{xx} - \omega^2 \frac{1}{c^2(x)} u = 0\\
u(0) = 1\\
i \omega u(1) + c(1) u_x(1) = 0
\end{cases}
\]
Этому приближению соответствует случай больших $\omega$ (более конкретно,
$\omega \gg \max |c_x(x)|$). Представим производную фазы функции $u$ в виде асимптотического
ряда по степеням $\omega$:
\begin{gather*}
u = C \exp \left\{ \int_0^x \Phi(\xi) d\xi \right\}\\
\Phi(x) = i\omega \sum_{k=0}^\infty (i\omega)^{-k}S_k(x)
\end{gather*}

Подставляя $u = C e^{\int_0^x \Phi(\xi) d\xi}$ в уравнение, получаем:
\[
\Phi'(x) + \Phi^2(x) = -\frac{\omega^2}{c^2(x)}.
\]

В нулевом приближении $\Phi_0(x) = i\omega S_0(x)$, откуда $S_0(x) =
\pm\dfrac{1}{c(x)}$.

В первом приближении $\Phi_1(x) = i\omega S_0(x) + S_1(x)$. Приравнивая
нулю члены порядка $\omega$, получаем
\[
i\omega S_0'(x) + 2 i \omega S_0 S_1 = 0 \Rightarrow S_1(x) =
-\frac{S_0'(x)}{2S_0(x)}
\]

Подставляя $\Phi_1(x)$ в представление $u$, получаем
\begin{multline}
u(x) \approx 
C \exp{\left\{
\int_0^x i\omega S_0(\xi) + S_1(\xi) d\xi
\right\}}
= C \exp \left\{ \pm\int_0^x \frac{i\omega d\xi}{c(\xi)} -\frac{1}{2} \ln
\frac{S_0(x)}{S_0(0)} \right\} = \\
= \frac{C \exp{\left\{
	\pm\int_0^x \frac{i\omega d\xi} {c(\xi)}
	\right\}}}{\sqrt{S_0(x)/S_0(0)}} =
\tilde C \sqrt{\frac{c(x)}{c(0)}} \exp{\left\{\pm \int_0^x \frac{i\omega d\xi}{c(\xi)} \right\}}
\label{eq:approx}
\end{multline}

Граничному условию в точке $x=1$ \eqref{eq:approx} удовлетворяет при знаке $+$,
а константа $C$ равна $1$ в силу условия в точке $x = 0$.
\[
u(x) \approx \sqrt\frac{c(x)}{c(0)} \exp\left\{\int_0^x \frac{i\omega
d\xi}{c(\xi)}\right\}
\]
Это решение достаточно хорошо описывает поведение реального решения уже даже при
$\omega = 5$.
\section{Метод прогонки}

Будем искать решение в виде прогоночного соотношения $u_n = P_n u_{n+1} + Q_n$. 
Сравнивая его с первым уравнением 
\[
b_1 u_1 + a_2 u_2 = d_1 \Rightarrow u_1 = -\frac{a_2}{b_1} u_2 +
\frac{d_1}{b_1},
\]
получаем значения для $P_1, Q_1$:
\begin{equation}
P_1 = -\frac{a_2}{b_1}, \qquad Q_1 = \frac{d_1}{b_1}.
\label{eq:start}
\end{equation}
Подставим соотношение $u_n = P_n u_{n+1} + Q_n$ в уравнение
\begin{gather*}
a_{n+1} u_n + b_{n+1} u_{n+1} + a_{n+2} u_{n+2} = d_{n+1}\\
a_{n+1} P_n u_{n+1} + a_{n+1} Q_n + b_{n+1} u_{n+1} + a_{n+2} u_{n+2} = d_{n+1}\\
(a_{n+1} P_n + b_{n+1}) u_{n+1} + a_{n+2} u_{n+2} = d_{n+1} - a_{n+1} Q_n\\
u_{n+1} = -\frac{a_{n+2}}{a_{n+1} P_n + b_{n+1}} u_{n+2} + \frac{d_{n+1} -
a_{n+1} Q_n}{a_{n+1} P_n + b_{n+1}}
\end{gather*}
Отсюда получаем соотношения для перехода от значений $P_n, Q_n$ к значениям
$P_{n+1}, Q_{n+1}$:
\begin{equation}
P_{n+1} = -\frac{a_{n+2}}{a_{n+1} P_n + b_{n+1}} \qquad
Q_{n+1} = \frac{d_{n+1} - a_{n+1} Q_n}{a_{n+1} P_n + b_{n+1}}
\label{eq:fwdstep}
\end{equation}
Таким способом можно вычислить все значения $P_i, Q_i$ при $i = 1, \dots, N$.
Рассмотрим последнее соотношение:
\[
u_N = P_N u_{N+1} + Q_N.
\]
Поскольку из \eqref{eq:fwdstep} получается $P_N = 0$, значение $u_N$ просто равно $Q_N$.
Теперь можно найти все остальные значения, пользуясь уже известным прогоночным
соотношением:
\begin{align}
&u_N = Q_N\\
&u_i = P_i u_{i+1} + Q_i, \qquad i = N-1, \dots, 1
\end{align}

Данный алгоритм существенно последовательный. Коэффициенты $P, Q$ зависят от
предыдущих, и могут вычисляться только друг за другом. Аналогично, при
вычислении решения каждое значение зависит от следующего и может вычисляться
только в обратном порядке.

\section{Метод редукции}

Для удобства будем нумеровать уравнения и неизвестные начиная с 0.

\[
\begin{pmatrix}
b_0 & c_0 \\
a_1 & b_1 & c_1 \\
& a_2 & b_2 & c_2 \\
&& \ddots & \ddots & \ddots\\
&&& a_{N-2} & b_{N-2} & c_{N-2}\\
&&&& a_{N - 1} & b_{N - 1}
\end{pmatrix}
\begin{pmatrix}
u_0\\u_1\\u_2\\\vdots\\u_{N-2}\\u_{N-1}
\end{pmatrix} = 
\begin{pmatrix}
d_0\\d_1\\d_2\\\vdots\\d_{N-2}\\d_{N-1}
\end{pmatrix}
\]

Чтобы использовать данный метод нужно дополнить систему до системы размера
$K + 1 = 2^k + 1 \geq N > K / 2 + 1$. Для этого можно положить 
\[a_i = 0,\, b_i = 1,\, c_i = 0,\, d_i = 0, \quad i = N, \dots, K\]

В основе алгоритма лежит процедура исключения из системы каждой нечетной
неизвестной, при этом размер системы уменьшается вдвое, а сама она остается
трехдиагональной (каждое уравнение содержит три соседние \emph{четные}
неизвестные).

Возьмем три последовательных уравнения --- с номерами $2m-1, 2m$ и $2m+1$:
\[
\begin{array}{c@{\,}c@{\,}c@{\,}c@{\,}c@{\,}c@{\,}c@{\,}c@{\,}cl}
a_{2m-1} u_{2m-2} &+& b_{2m-1} u_{2m-1} &+& c_{2m-1} u_{2m} &&&&& = d_{2m-1}\\
&&a_{2m} u_{2m-1} &+& b_{2m} u_{2m} &+& c_{2m} u_{2m+1} &&& = d_{2m}\\
&&&&a_{2m+1} u_{2m} &+& b_{2m+1} u_{2m+1} &+& c_{2m+1} u_{2m+2} & = d_{2m+1}
\end{array}
\]
Мы хотим преобразовать их к виду
\[
\begin{array}{c@{\,}c@{\,}c@{\,}c@{\,}c@{\,}c@{\,}c@{\,}c@{\,}cl}
a_{2m-1} u_{2m-2} &+& b_{2m-1} u_{2m-1} &+& c_{2m-1} u_{2m} &&&&& = d_{2m-1}\\
a'_{2m} u_{2m-2} &&+&& b'_{2m} u_{2m} &&+&& c'_{2m} u_{2m+2} & = d'_{2m}\\
&&&&a_{2m+1} u_{2m} &+& b_{2m+1} u_{2m+1} &+& c_{2m+1} u_{2m+2} & = d_{2m+1}
\end{array}
\]
Нечетные уравнения оставлены, чтобы была возможность восстановить значения
нечетных неизвестных. 
\begin{gather}
\nonumber
b'_{2m} = b_{2m} - \frac{a_{2m}c_{2m-1}}{b_{2m-1}} -
\frac{c_{2m}a_{2m+1}}{b_{2m+1}}\\\label{eq:round}
d'_{2m} = d_{2m} - \frac{a_{2m}d_{2m-1}}{b_{2m-1}} -
\frac{c_{2m}d_{2m+1}}{b_{2m+1}}\\\nonumber
a'_{2m} = -\frac{a_{2m}a_{2m-1}}{b_{2m-1}}\\\nonumber
c'_{2m} = -\frac{c_{2m}c_{2m+1}}{b_{2m+1}}
\end{gather}

Предположим, что мы уже проредили неизвестные, оставив только каждую $s=2^p$
неизвестную. Тогда формулы \eqref{eq:round} преобразуются к виду
\begin{gather}
\nonumber
b'_{2sm} = b_{2sm} - \frac{a_{2sm}c_{2sm-s}}{b_{2sm-s}} -
\frac{c_{2sm}a_{2sm+s}}{b_{2sm+s}}\\\label{eq:rounds}
d'_{2sm} = d_{2sm} - \frac{a_{2sm}d_{2sm-s}}{b_{2sm-s}} -
\frac{c_{2sm}d_{2sm+s}}{b_{2sm+s}}\\\nonumber
a'_{2sm} = -\frac{a_{2sm}a_{2sm-s}}{b_{2sm-s}}\\\nonumber
c'_{2sm} = -\frac{c_{2sm}c_{2sm+s}}{b_{2sm+s}}
\end{gather}

Далее, когда от системы уравнений остается только два уравнения с двумя
неизвестными, она решается явно:
\begin{gather*}
b_0 u_0 + c_K u_K = d_0\\
a_0 u_0 + b_K u_K = d_K\\
u_0 = \frac{d_0 b_K - c_K d_K}{b_0 b_K - a_0 c_K}\\
u_K = \frac{b_0 d_K - a_0 d_0}{b_0 b_K - a_0 c_K}
\end{gather*}

Теперь $u_{K/2}$ можно найти из уравнения
\begin{gather*}
a_{K/2} u_0 + b_{K/2} u_{K/2} + c_{K/2} u_K = d_{K/2}\\
u_{K/2} = \frac{d_{K/2} - a_{K/2}u_0 - c_{K/2} u_K}{b_{K/2}}
\end{gather*}

Аналогичным образом, найдя значения неизвестных с шагом $s = 2^p$, пользуясь
уравнениями с номерами $ms + s/2$ восстанавливаются значения неизвестных с шагом
$s = 2^{p-1}$.

Алгоритм имеет ту же сложность $O(N)$, что и алгоритм прогонки, однако, в отличие
от алгоритма прогонки, алгоритм редукции достаточно хорошо параллелится,
поскольку его основные операции --- одновременное сдваивание коэффициентов и
одновременное вычисление решения во всех серединах. Этот алгоритм используется
на массивно-параллельных архитектурах (видеокартах). К тому же, метод более
устойчив к вычислительным ошибкам, нежели алгоритм прогонки, хотя оба они
устойчивы, если матрица имеет диагональное преобладание. В нашей задаче
диагонального преобладания нет.

\floatname{algorithm}{Алгоритм}
\begin{algorithm}
\caption{Метод редукции для решения трехдиагональной системы}
\label{alg:reduct}
\begin{algorithmic}[1]
\State $K \Leftarrow 2^{\lceil \log_2 (N-1) \rceil}$
\For {$j = N,\, N+1,\, \dots, K$}
	\State $a_j \Leftarrow 0$ \Comment Дополнить матрицу до подходящего размера
	\State $c_j \Leftarrow 0$
	\State $d_j \Leftarrow 0$
	\State $b_j \Leftarrow 1$
\EndFor
\For {$s = 1,\, 2,\, 4,\, \dots,\, K/2$}
	\State $C \Leftarrow -\dfrac{c_0}{b_s}$
	\Comment Обработать нулевое уравнение отдельно
	\State $b_0 \Leftarrow b_0 + C a_s$
	\State $d_0 \Leftarrow d_0 + C d_s$
	\State $c_0 \Leftarrow C c_s$

	\State $A \Leftarrow -\dfrac{a_K}{b_{K-s}}$
	\Comment Обработать последнее уравнение отдельно
	\State $b_K \Leftarrow b_K + A c_{K-s}$
	\State $d_K \Leftarrow d_K + A d_{K-s}$
	\State $a_K \Leftarrow A a_{K-s}$

	\For {$j = 2s,\, 4s,\, 6s,\, \dots,\, K - 2s$}
		\State $A \Leftarrow -\dfrac{a_j}{b_{j-s}}$
		\State $C \Leftarrow -\dfrac{c_j}{b_{j+s}}$
		\State $b_j \Leftarrow b_j + A c_{j - s} + C a_{j + s}$ 
		\State $d_j \Leftarrow d_j + A d_{j - s} + C d_{j + s}$
		\State $a_j \Leftarrow A a_{j-s}$
		\State $c_j \Leftarrow C c_{j+s}$
	\EndFor
\EndFor
\State $u_0 = \dfrac{d_0 b_K - c_K d_K}{b_0 b_K - a_0 c_K}$
\Comment Найти крайние неизвестные из системы $2 \times 2$
\State $u_K = \dfrac{b_0 d_K - a_0 d_0}{b_0 b_K - a_0 c_K}$
\For {$s = K/2,\, K/4,\, \dots, 4, 2, 1$}
	\For {$j = s,\, 3s,\, 5s,\, \dots,\, K - s$}
		\State $u_j = \dfrac{d_j - a_j u_{j-s} - c_j u_{j+s}}{b_j}$
	\EndFor
\EndFor
\end{algorithmic}
\end{algorithm}


\end{document}
